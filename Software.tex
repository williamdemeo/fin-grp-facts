\GAP\ contains a library of primitive permutation groups which includes the
following permutation groups up to permutation isomorphism (i.e., up to
conjugacy in the corresponding symmetric group):
\begin{itemize}
\item 
all primitive permutation groups of degree $< 2500$, calculated by Colva Roney-Dougal in~\cite{Roney-Dougal:2005}; in particular,
\item the primitive permutation groups up to degree 50, calculated by C. Sims,
\item the primitive groups with insoluble socles of degree $< 1000$ as calculated in~\cite{Dixon:1988},
\item the solvable (hence affine) primitive permutation groups of degree $< 256$ as calculated in~\cite{Short:1992},
\item some insolvable affine primitive permutation groups of degree $< 256$ as calculated in~\cite{Theissen:1997}.
\item The solvable primitive groups of degree up to 999 as calculated in~\cite{Eick:2002}.
\item The primitive groups of affine type of degree up to 999 as calculated in~\cite{Roney-Dougal:2003}.
\end{itemize}
%Not all groups are named, those which do have names use ATLAS notation. Not all names are necessary unique!
The list given in~\cite{Roney-Dougal:2005} is believed to be complete,
correcting various omissions in~\cite{Dixon:1988}, \cite{Short:1992},
and~\cite{Theissen:1997}.  (The latest version of \GAP, version 4.5, is still in
beta-testing, but we expect it will incorporate the results
of Coutts, Quick, and Roney-dougal~\cite{Coutts_theprimitive} cataloging all
primitive groups of degree $< 4096$.)

%% Section 41.1 of the \gap\ manual covers primitive permutation groups.  Here are some
%% points that are important for our work.
\begin{itemize}
\item 
{\tt ONanScottType( $G$ )}
returns the type of a primitive permutation group $G$, according to the O'Nan-Scott classification. The
labeling of the different types is not consistent in the literature, and
\gap\ uses the following:
\begin{itemize}
\item 1 Affine.
\item 2 Almost simple.
\item 3a Diagonal, Socle consists of two normal subgroups.
\item 3b Diagonal, Socle is minimal normal.
\item 4a Product action with the first factor primitive of type 3a.
\item 4b Product action with the first factor primitive of type 3b.
\item 4c Product action with the first factor primitive of type 2.
\item 5 Twisted wreath product.
\end{itemize}
As it can contain letters, the type is returned as a string.
If $G$ is not a permutation group or does not act primitively on the points moved by it, the result is undefined.
Some examples of primitive permutation groups of relatively
small degree that are of types 2, 3a, 3b, and 4c are given below in Section~\ref{sec:examples-onan-scott}.
(Note, that for groups of degree up to 2499, O'Nan-Scott types 4a, 4b and 5
cannot occur.)


\item {\tt SocleTypePrimitiveGroup( $G$ )}
returns the socle type of a primitive permutation group. The socle of a primitive group is the direct product
of isomorphic simple groups, therefore the type is indicated by a record with
components {\tt series, parameter} 
(both as described under IsomorphismTypeInfoFiniteSimpleGroup, see below) and width for the
number of direct factors.
If $G$ does not have a faithful primitive action, the result is undefined.

\item\protect\footnotemark
\footnotetext{See Section 37.15 of the \gap\ manual.}
{\tt IsomorphismTypeInfoFiniteSimpleGroup( $G$ )}
For a finite simple group $G$, IsomorphismTypeInfoFiniteSimpleGroup returns a record with components
series, name and possibly parameter, describing the isomorphism type of $G$. The component name is a
string that gives name(s) for $G$, and series is a string that describes the following series.
(If different characterizations of $G$ are possible only one is given by series and parameter, while name may
give several names.)
{\footnotesize
\begin{itemize}
\item 
``A'' Alternating groups, parameter gives the natural degree.
\item 
``L'' Linear groups (Chevalley type A), parameter list [n,q] indicates L(n, q).
\item 
``2A'' Twisted Chevalley type 2 A, parameter list [n,q] indicates 2 A(n, q).
\item 
``B'' Chevalley type B, parameter list [n,q] indicates B (n, q).
\item 
``2B'' Twisted Chevalley type 2 B, parameter q indicates 2 B (2, q).
\item 
``C'' Chevalley type C, parameter list [n,q] indicates C (n, q).
\item 
``D'' Chevalley type D, parameter list [n,q] indicates D(n, q).
\item 
``2D'' Twisted Chevalley type 2 D, parameter list [n,q] indicates 2 D(n, q).
\item 
``3D'' Twisted Chevalley type 3 D, parameter q indicates 3 D(4, q).
\item 
``E'' Exceptional Chevalley type E, parameter list [n,q] indicates En (q). \\The value of n is 6, 7 or 8.
\item 
``2E'' Twisted exceptional Chevalley type E6, parameter q indicates 2 E6 (q).
\item 
``F'' Exceptional Chevalley type F , parameter q indicates F (4, q).
\item 
``2F'' Twisted exceptional Chevalley type 2 F (Ree groups), parameter q indicates 2 F (4, q).
\item 
``G'' Exceptional Chevalley type G, parameter q indicates G(2, q).
\item 
``2G'' Twisted exceptional Chevalley type 2 G (Ree groups), parameter q indicates 2 G(2, q).
\item 
``Spor'' Sporadic groups, name gives the name.
\item 
``Z'' Cyclic groups of prime size, parameter gives the size.
\end{itemize}
}
An equal sign in the name denotes different naming schemes for the same group, a tilde sign abstract
isomorphisms between groups constructed in a different way.

\end{itemize}
