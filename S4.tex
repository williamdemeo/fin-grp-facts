
The group $S_4$ contains the following permutations:
\begin{center}
\begin{tabular}{|c|c|}
\hline
permutations & type\\[4pt]
\hline
(01), (02), (03), (12), (13), (23) & 2-cycles \\[4pt]
(01)(23), (02)(13), (03)(12) & product of 2-cycles \\[4pt]
(012), (013), (021), (023), (031), (032), (123), (132) &  3-cycles \\[4pt]
(0123), (0132), (0213), (0231), (0312), (0321) & 4-cycles \\[4pt]
\hline
\end{tabular}
\end{center}

Within each row of the table, the elements are listed in lexicographic order.  We might
use this ordering as a guide when assigning the labels $\{0, 1, 2, \ldots, 23\}$ to
the elements of $S_4$.  For example, one possible labelling 
(that assigns even numbers to elements of $A_4$) is shown here.
\\

{\scriptsize
\begin{tabular}{|l|c|c|c|c|c|c|c|c|c|}
\hline
label& 0&1& 2& 3& 4& 5& 6& 7& 8\\
\hline
element&
 e &
(01)&
(01)(23)&
(02)&
(02)(13)&
(03)&
(03)(12)&
(12)&
(012)\\
\hline
\end{tabular}
}

{\scriptsize
\begin{tabular}{|l|c|c|c|c|c|c|c|c|}
\hline
label & 9& 10& 11& 12& 13& 14& 15& 16\\
\hline
element &(13)&(013)& (23)&(021)&(0123)&(023)&(0132)&(031)\\
\hline
\end{tabular}
}

{\scriptsize
\begin{tabular}{|l|c|c|c|c|c|c|c|}
\hline
label& 17& 18& 19& 20& 21& 22& 23\\
\hline
element &(0213)&(032)&(0231)&(123)&(0312)&(132)&(0321)\\
\hline
\end{tabular}
}

\medskip

There are 30 subgroups of $S_4$. The following table presents
them all, except $(e)$ and $S_4$, and their element labels:

{\scriptsize
\begin{center}
\begin{tabular}{|r|c|c|c|c|}
\hline
&name & elements & element labels & order\\
\hline
0& $A_4$ & $\{e, (01)(23), (02)(13), (03)(12),(012), (013), $& $\{0,2, 4, 6,\ldots, 22\}$ & 12\\[4pt]
 &       &  \phantom{XXX}$ (021), (023), (031), (032), (123), (132)\}$  &  &\\[4pt]
\hline
1&$V_4$ & $\{e, (01)(23), (02)(13), (03)(12)\}$ & $\{0,2,4,6\}$& 4\\[4pt]
\hline
2&$v_i$ & $\{e, (01)(23)\},\; \{e, (02)(13)\}, \; \{e, (03)(12)\}$ & $\{0,2\},
\{0,4\}, \{0,6\}$& 2, 2, 2\\[4pt]
\hline
5& $P_0$ & $\{e, (012), (021)\}$ &$\{0,8,12\}$& 3\\[4pt]
\hline
6&$P_1$ & $\{e, (013), (031)\}$ &$\{0,10,16\}$& 3\\[4pt]
\hline
7&$P_2$ & $\{e, (023), (032)\}$ &$\{0,14,18\}$& 3\\[4pt]
\hline
8&$P_3$ & $\{e, (123), (132)\}$ &$\{0,20,22\}$& 3\\[4pt]
\hline
9&$D$ & $\{e, (01),(01)(23), (02)(13),$ & $\{0,1,2,4,6,11,17,21\}$& 8\\[4pt]
& & \phantom{XXX} $(03)(12), (23), (0213), (0312)\}$ & & \\[4pt]
\hline
10&$d$ & $\{e, (01)(23), (0213), (0312)\}$ & $\{0,2,17,21\}$& 4\\[4pt]
\hline
11&$D'$ & $\{e, (01)(23), (02), (02)(13), $&$\{0,2,3,4,6,9,13,23\}$& 8\\[4pt]
& & \phantom{XXX} $(03)(12), (13), (0123), (0321)\}$     && \\[4pt]
\hline
12&$d'$ & $\{e, (02)(13), (0123), (0321)\}$ &$\{0,4,13,23\}$& 4\\[4pt]
\hline
13&$D''$ & $\{e, (01)(23), (02)(13), (03), $ &$\{0,2,4,5,6,7,15,19\}$& 8\\[4pt]
& &  \phantom{XXX} $(03)(12), (12), (0132), (0231)\}$ &&\\[4pt]
\hline
14&$d''$ & $\{e, (03)(12), (0132), (0231)\}$ &$\{0,6,15,19\}$& 4\\[4pt]
\hline
15&$H_0$ & $\{e, (01), (02), (12), (012), (021)\}$ & $\{0, 1, 3, 7, 8, 12\}$& 6\\[4pt]
\hline
16&$H_1$ & $\{e, (01), (03), (13), (013), (031)\}$ & $\{0, 1, 5, 9, 10, 16\}$& 6\\[4pt]
\hline
17&$H_2$ & $\{e, (02), (03), (23), (023), (032)\}$ & $\{0, 3, 5, 11, 14, 18\}$& 6\\[4pt]
\hline
18&$H_3$ & $\{e, (12), (13), (23), (123), (132)\}$ & $\{0, 7, 9, 11, 20, 22\}$& 6\\[4pt]
\hline
19&$A$ & $\{e, (01),(01)(23),(23) \}$ & $\{0, 1, 2, 11\}$& 4 \\[4pt]
\hline
20&$a_i$ & $\{e, (01)\}, \; \{e, (23) \}$ & $\{0, 1\}, \;\{0, 11\}$& 2, 2 \\[4pt]
\hline
22&$B$ & $\{e, (02), (02)(13),(13)\}$ & $\{0, 3, 4, 9\}$& 4 \\[4pt]
\hline
23&$b_i$ & $\{e, (02)\}, \; \{e, (13)\}$ & $\{0, 3\}, \, \{0, 9\}$& 2, 2 \\[4pt]
\hline
25&$C$ & $\{e, (03), (03)(12), (12)\}$ & $\{0, 5, 6, 7\}$& 4 \\[4pt]
\hline
26&$c_i$ & $\{e, (03)\}, \; \{e, (12)\}$ & $\{0, 5\}, \, \{0, 7\}$& 2, 2 \\[4pt]
\hline
\end{tabular}
\end{center}
}

Let $X = \{0, 1, 2, \ldots, 23\}$.  
If you stare long enough at the table of subgroups on the previous page, it becomes apparent
how we can construct partitions in $\Eq(X)$ using cosets of the appropriate subgroups
in such a way that the resulting equivalences generate a sublattice isomorphic to
$\Eq(4)$. Specifically, look at $A, B, C$, and $H_i \; (i=0,1,2,3)$, and the
subgroups below them, and notice how their intersections resemble the meet 
structure of $\Eq(4)$.  

Let $\delta_i$ be the equivalence constructed from the cosets of $H_i$.  From the
previous observation concerning the meet structure, it is clear that the
four equivalences $\delta_i \; (i=0,1,2,3)$ will serve as the four generating
coatoms of the $\Eq(4)$ sublattice.  Thus, we need only construct these four
equivalences, and let them generate the rest of the $\Eq(4)$ sublattice.
(Though we might verify that three of the coatoms they generate correspond to
the equivalences we would get using cosets of $A, B$, and $C$.)

The cosets of $H_i$, and the resulting partitions $\delta_i$, are shown below.
%% ~\\
%% \vspace{-15mm}
{\small
  \[
H_0 = \{e, (01), (02), (12), (012), (021)\} = \{0, 1, 3, 7, 8, 12\}
\]
\[
H_0(03) = \{(03), (031), (032), (03)(12), (0312), (0321)\} = \{5,6,16,18,21,23\}
\]
\[
H_0(13) = \{(13), (013), (02)(13), (132), (0132), (0213)\} = \{4,9,10,15,17,22\}
\]
\[
H_0(23) = \{(23), (01)(23), (023), (123), (0123), (0231)\} = \{2,11,13,14,20,19\}
\]
Define $\delta_0 = (H_0|H_0(03)|H_0(13)|H_0(23))$. That is,
\[
\delta_0 = (0, 1, 3, 7, 8, 12|5,6,16,18,21,23|4,9,10,15,17,22|2,11,13,14,20,19)
\]
Similarly,
\[
H_1 = \{e, (01), (03), (13), (013), (031)\} = \{0, 1, 5, 9, 10, 16\}
\]
\[
H_1(02) = \{(02), (021), (023), (02)(13), (0213), (0231)\} = \{3, 4, 12, 14, 17, 19\}
\]
\[
H_1(12) = \{(12), (012), (03)(12), (123), (0123), (0312)\} = \{6, 7, 8, 13, 20, 21\}
\]
\[
H_1(23) = \{(23), (01)(23), (032), (132), (0132), (0321)\} = \{2, 11, 15, 18, 22, 23\}
\]
\[
\delta_1 = (0, 1, 5, 9, 10, 16|3, 4, 12, 14, 17, 19|6, 7, 8, 13, 20, 21|2, 11, 15, 18, 22, 23)
\]
\[
H_2 = \{e, (02), (03), (23), (023), (032)\} = \{0, 3, 5, 11, 14, 18\}
\]
\[
H_2(01) = \{(01), (012), (013), (01)(23), (0123), (0132)\} = \{1,2,8,10,13,15\}
\]
\[
H_2(12) = \{(12), (021), (03)(12), (132), (0213), (0321)\} = \{6,7,12,17,22,23\}
\]
\[
H_2(13) = \{(13), (02)(13), (031), (123), (0231), (0312)\} = \{4,9,16,20,19,21\}
\]
\[
\delta_2 = (0,3,5,11,14,18|1,2,8,10,13,15|6,7,12,17,22,23|4,9,16,20,19,21)
\]
\[
H_3 = \{e, (12), (13), (23), (123), (132)\} = \{0, 7, 9, 11, 20, 22\}
\]
\[
H_3(01) = \{(01), (021), (031), (01)(23), (0231), (0321)\} = \{1,2,12,16,19,23\}
\]
\[
H_3(02) = \{(02), (012), (02)(13), (032), (0312), (0132)\} = \{3,4,8,15,18,21\}
\]
\[
H_3(03) = \{(03), (03)(12), (013), (023), (0123), (0213)\} = \{5,6,10,13,14,17\}
\]
\[
\delta_3 = (0,7,9,11,20,22|1,2,12,16,19,23|3,4,8,15,18,21|5,6,10,13,14,17).
\]
}

\newpage
Let $L\leq \Eq(X)$ denote the lattice generated by $\delta_i\;(i=0,1,2,3)$. 
Our computer programs identify the unary operations on $X$ which respect all four 
equivalences $\delta_i\;(i=0,1,2,3)$.  There are 48 such operations, 
so $|\lambda(L)| = 48$.  Leaving out the 24 constant maps, the
remaining 23 operations appear below:
{\small
\begin{verbatim}
 0  1  2  3  4  5  6  7  8  9 10 11 12 13 14 15 16 17 18 19 20 21 22 23
 1  0 11  8 21 10 17 12  3 16  5  2  7 14 13 18  9  6 15 20 19  4 23 22
 2 11  0 13  6 15  4 19 14 23 18  1 20  3  8  5 22 21 10  7 12 17 16  9
 3 12 23  0  9 14 13  8  7  4 17 18  1  6  5 22 19 10 11 16 21 20 15  2
 4 17  6  9  0 19  2 15 22  3 12 21 10 23 16  7 14  1 20  5 18 11  8 13
 5 16 19 18 15  0  7  6 21 10  9 14 23 20 11  4  1 22  3  2 13  8 17 12
 6 21  4 23  2  7  0  5 16 13 20 17 18  9 22 19  8 11 12 15 10  1 14  3
 7  8 15 12 19  6  5  0  1 20 13 22  3 10 17  2 21 14 23  4  9 16 11 18
 8  7 22  1 16 13 14  3 12 21  6 15  0 17 10 23 20  5  2  9  4 19 18 11
 9 10 13  4  3 16 23 22 15  0  1 20 17  2 19  8  5 12 21 14 11 18  7  6
10  9 20 15 18  1 12 17  4  5 16 13 22 19  2 21  0 23  8 11 14  3  6  7
11  2  1 14 17 18 21 20 13 22 15  0 19  8  3 10 23  4  5 12  7  6  9 16
12  3 18  7 20 17 10  1  0 19 14 23  8  5  6 11  4 13 22 21 16  9  2 15
13 20  9  2 23  8  3 14 19  6 21 10 11  4 15 16  7 18  1 22 17 12  5  0
14 19 16 11 22  3  8 13 20 17  4  5  2 21 18  9 12 15  0 23  6  7 10  1
15 22  7 10  5  2 19  4 17 18 23  8  9 12  1  6 11 16 13  0  3 14 21 20
16  5 14 21  8  9 22 23 18  1  0 19  6 11 20  3 10  7  4 13  2 15 12 17
17  4 21 22 11 12  1 10  9 14 19  6 15 16 23 20  3  2  7 18  5  0 13  8
18 23 12  5 10 11 20 21  6 15 22  3 16  7  0 17  2  9 14  1  8 13  4 19
19 14  5 20  7  4 15  2 11 12  3 16 13 18 21  0 17  8  9  6 23 22  1 10
20 13 10 19 12 21 18 11  2  7  8  9 14 15  4  1  6  3 16 17 22 23  0  5
21  6 17 16  1 20 11 18 23  8  7  4  5 22  9 12 13  0 19 10 15  2  3 14
22 15  8 17 14 23 16  9 10 11  2  7  4  1 12 13 18 19  6  3  0  5 20 21
23 18  3  6 13 22  9 16  5  2 11 12 21  0  7 14 15 20 17  8  1 10 19  4
\end{verbatim}
}
\newpage
The lattice $L\cong \Eq(4)$ has 15 elements, while the closure of $L$ in $\Eq(X)$ has
30 elements.  Besides the all relation and the zero relation, the closure of $L$
consists of the following congruences (the elements of $L$ are identified by *):

{\small
\begin{verbatim}
 |0,1,2,4,6,11,17,21|3,9,10,12,13,18,20,23|5,7,8,14,15,16,19,22| D
*|0,1,2,11|3,12,18,23|4,6,17,21|5,14,16,19|7,8,15,22|9,10,13,20| A
*|0,1,3,7,8,12|2,11,13,14,19,20|4,9,10,15,17,22|5,6,16,18,21,23| H_0
*|0,1,5,9,10,16|2,11,15,18,22,23|3,4,12,14,17,19|6,7,8,13,20,21| H_1
*|0,1|2,11|3,12|4,17|5,16|6,21|7,8|9,10|13,20|14,19|15,22|18,23| a_0
 |0,2,3,4,6,9,13,23|1,8,11,14,16,17,21,22|5,7,10,12,15,18,19,20| D'
 |0,2,4,6,8,10,12,14,16,18,20,22|1,3,5,7,9,11,13,15,17,19,21,23| A_4
 |0,2,4,5,6,7,15,19|1,10,11,12,17,18,20,21|3,8,9,13,14,16,22,23| D''
 |0,2,4,6|1,11,17,21|3,9,13,23|5,7,15,19|8,14,16,22|10,12,18,20| V_4
 |0,2,17,21|1,4,6,11|3,10,20,23|5,8,19,22|7,14,15,16|9,12,13,18| d
 |0,2|1,11|3,23|4,6|5,19|7,15|8,22|9,13|10,20|12,18|14,16|17,21| v_0
*|0,3,5,11,14,18|1,2,8,10,13,15|4,9,16,19,20,21|6,7,12,17,22,23| H_2
*|0,7,9,11,20,22|1,2,12,16,19,23|3,4,8,15,18,21|5,6,10,13,14,17| H_3
*|0,11|1,2|3,18|4,21|5,14|6,17|7,22|8,15|9,20|10,13|12,23|16,19| a_1
*|0,3,4,9|1,8,16,21|2,6,13,23|5,10,15,18|7,12,19,20|11,14,17,22| B
*|0,3|1,8|2,13|4,9|5,18|6,23|7,12|10,15|11,14|16,21|17,22|19,20| b_0
 |0,8,12|1,3,7|2,14,20|4,10,22|5,21,23|6,16,18|9,15,17|11,13,19| P_0
 |0,4,13,23|1,14,21,22|2,3,6,9|5,12,15,20|7,10,18,19|8,11,16,17| d'
 |0,4|1,21|2,6|3,9|5,15|7,19|8,16|10,18|11,17|12,20|13,23|14,22| v_1
*|0,5,6,7|1,10,12,17|2,4,15,19|3,8,13,14|9,16,22,23|11,18,20,21| C
 |0,6,15,19|1,17,18,20|2,4,5,7|3,13,16,22|8,9,14,23|10,11,12,21| d''
 |0,6|1,17|2,4|3,13|5,7|8,14|9,23|10,12|11,21|15,19|16,22|18,20| v_2
*|0,9|1,16|2,23|3,4|5,10|6,13|7,20|8,21|11,22|12,19|14,17|15,18| b_1
*|0,5|1,10|2,15|3,14|4,19|6,7|8,13|9,16|11,18|12,17|20,21|22,23| c_0
 |0,10,16|1,5,9|2,18,22|3,17,19|4,12,14|6,8,20|7,13,21|11,15,23| P_1
 |0,14,18|1,13,15|2,8,10|3,5,11|4,16,20|6,12,22|7,17,23|9,19,21| P_2
*|0,7|1,12|2,19|3,8|4,15|5,6|9,22|10,17|11,20|13,14|16,23|18,21| c_1
 |0,20,22|1,19,23|2,12,16|3,15,21|4,8,18|5,13,17|6,10,14|7,9,11| P_3
\end{verbatim}
}
\newpage
The cosets of the subgroups $A, B$, and $C$ should\footnote{I haven't checked carefully that these are actually the cosets.}
correspond to the following partitions:
\[
\alpha = (0,1,2,11|3,12,18,23|4,6,17,21|5,14,16,19|7,8,15,22|9,10,13,20)
\]
\[
\beta = (0,3,4,9|1,8,16,21|2,6,13,23|5,10,15,18|7,12,19,20|11,14,17,22)
\]
\[
\gamma = (0,5,6,7|1,10,12,17|2,4,15,19|3,8,13,14|9,16,22,23|11,18,20,21)
\]
These equivalences are the ``non-generating'' coatoms of $L$.  
They generate an $M_3$ sublattice, and each is above two atoms 
(unlike the ``generating'' coatoms $\delta_i$, each of which covers three atoms).  
The six atoms of $L$ are marked on the previous page by $a_i, b_i, c_i \, (i=0,1)$.  They are as follows:
{\scriptsize
\[
a_0 = \alpha \meet \delta_0 = \alpha\meet \delta_1 = (0,1|2,11|3,12|4,17|5,16|6,21|7,8|9,10|13,20|14,19|15,22|18,23)
\]
\[
a_1 = \alpha \meet \delta_2 = \alpha\meet \delta_3 = (0,11|1,2|3,18|4,21|5,14|6,17|7,22|8,15|9,20|10,13|12,23|16,19)
\]
\[
b_0 = \beta \meet \delta_0 = \beta \meet \delta_2 = (0,3|1,8|2,13|4,9|5,18|6,23|7,12|10,15|11,14|16,21|17,22|19,20)
\]
\[
b_1 =\beta \meet \delta_1 = \beta\meet \delta_3 = (0,9|1,16|2,23|3,4|5,10|6,13|7,20|8,21|11,22|12,19|14,17|15,18)
\]
\[
c_0 =\gamma \meet \delta_1 = \gamma \meet \delta_2 = (0,5|1,10|2,15|3,14|4,19|6,7|8,13|9,16|11,18|12,17|20,21|22,23)
\]
\[
c_1 =\gamma \meet \delta_0 = \gamma \meet \delta_3 = (0,7|1,12|2,19|3,8|4,15|5,6|9,22|10,17|11,20|13,14|16,23|18,21)
\]
}
We could try to prove that the closure of such
``uniform'' embeddings of $\Eq(n)$ in $\Eq(n!)$ should always contain the congruence 
corresponding to the cosets of $A_n$. 
In light of the foregoing data, we might try to prove some or all of the
following, for every uniform embedding of $\Eq(4)\cong L \leq \Eq(X)$:
\begin{enumerate}[(i)]
\item $|X|\geq 24$
\item $\forall \alpha \in L \setminus \{1\}$ the block size of $\alpha$ is strictly
  less than $|X|/2$
\item $L$ does not contain the partition corresponding to cosets of $A_n$
\item the closure of $L$ contains the partition corresponding to cosets of $A_n$
\item the closure of $L$ contains an element with block size $|X|/2$
\end{enumerate}
Item (i) is easily within reach, and (ii) and (iii) seem plausible. 
Of course, % (ii) implies (iii), and (iv) implies (v).  
if both (ii) and (v) hold, or both (iii) and (iv) hold, the FLRP is solved.
